We explain  a state machine by deconstructing the decade counter

The block diagram of a decade counter (repeatedly counts up from 0 to 9)
is available in \figref{fig:decade_counter}.  The {\em incrementing } decoder
and {\em display} decoder are part of {\em combinational} logic, while
the {\em delay} is part of {\em sequential} logic.
%
%
\begin{enumerate}[label=\arabic*.,ref=\theenumi]
%
\item \figref{fig:fsm_counter} shows a {\em finite state machine} (FSM) diagram for the decade counter in Fig \ref{fig:decade_counter}  $s_0$ is the state when the input to the incrementing decoder is 0.  The {\em state transition table} for the FSM is Table \ref{tab:ide/7447/counter_decoder},
%	0 in \cite{gvv_kmap}
		where the present state is denoted by the variables $W,X,Y,Z$ and the next state by $A,B,C,D$.  
\begin{figure}[H]
%\begin{figure}[!ht]
\centering
\resizebox {\columnwidth} {!} {
\input{ide/fsm/figs/fsm_counter}
}
\caption{FSM for the decade counter}
\label{fig:fsm_counter}
\end{figure}
\item The FSM implementation is available in Fig \ref{fig:dff}  The {\em flip-flops} hold the input for the time that is given by the {\em clock}  This is nothing but the implementation of the {\em Delay} block in Fig \ref{fig:decade_counter}
%
\begin{figure}[H]
\resizebox {\columnwidth} {!} {
\input{ide/fsm/figs/dff}
}
\caption{Decade counter FSM implementation using D-Flip Flops}
\label{fig:dff}
\end{figure}
%
\item The hardware cost of the system is given by
\begin{equation}
\text{No of D Flip-Flops} = \myceil{\log_{2}\brak{\text{No of States}}}
\end{equation}
For the FSM in Fig \ref{fig:fsm_counter}, the number of states is 9, hence the number of flip flops required = 4  
\item Draw the state transition diagram for 
a decade down counter (counts from 9 to 0 repeatedly) using an FSM.  
\item Write the state transition table for the down counter.
\item Obtain the state transition equations with and without don't cares.
\item Verify your design using an arduino.
%\item Repeat the above exercises by designing a circuit that can detect 3 consecutive s in a bitstream 
\end{enumerate}


